

%% This file was auto-generated by IPython.
%% Conversion from the original notebook file:
%%
\documentclass[11pt,english]{article}

%% This is the automatic preamble used by IPython.  Note that it does *not*
%% include a documentclass declaration, that is added at runtime to the overall
%% document.

\usepackage{amsmath}
\usepackage{amssymb}
\usepackage{graphicx}
\usepackage{grffile}
\usepackage{ucs}
\usepackage[utf8x]{inputenc}

% Scale down larger images
\usepackage[export]{adjustbox}

%fancy verbatim
\usepackage{fancyvrb}
% needed for markdown enumerations to work
\usepackage{enumerate}

% Slightly bigger margins than the latex defaults
\usepackage{geometry}
\geometry{verbose,tmargin=3cm,bmargin=3cm,lmargin=2.5cm,rmargin=2.5cm}

% Define a few colors for use in code, links and cell shading
\usepackage{color}
\definecolor{orange}{cmyk}{0,0.4,0.8,0.2}
\definecolor{darkorange}{rgb}{.71,0.21,0.01}
\definecolor{darkgreen}{rgb}{.12,.54,.11}
\definecolor{myteal}{rgb}{.26, .44, .56}
\definecolor{gray}{gray}{0.45}
\definecolor{lightgray}{gray}{.95}
\definecolor{mediumgray}{gray}{.8}
\definecolor{inputbackground}{rgb}{.95, .95, .85}
\definecolor{inputbackground_1}{rgb}{.98, .98, 0.8}
\definecolor{outputbackground}{rgb}{.95, .95, .95}
\definecolor{traceback}{rgb}{1, .95, .95}

% new ansi colors
\definecolor{brown}{rgb}{0.54,0.27,0.07}
\definecolor{purple}{rgb}{0.5,0.0,0.5}
\definecolor{darkgray}{gray}{0.25}
\definecolor{lightred}{rgb}{1.0,0.39,0.28}
\definecolor{lightgreen}{rgb}{0.48,0.99,0.0}
\definecolor{lightblue}{rgb}{0.53,0.81,0.92}
\definecolor{lightpurple}{rgb}{0.87,0.63,0.87}
\definecolor{lightcyan}{rgb}{0.5,1.0,0.83}

% Framed environments for code cells (inputs, outputs, errors, ...).  The
% various uses of \unskip (or not) at the end were fine-tuned by hand, so don't
% randomly change them unless you're sure of the effect it will have.
\usepackage{framed}

% remove extraneous vertical space in boxes
\setlength\fboxsep{0pt}

% codecell is the whole input+output set of blocks that a Code cell can
% generate.

% TODO: unfortunately, it seems that using a framed codecell environment breaks
% the ability of the frames inside of it to be broken across pages.  This
% causes at least the problem of having lots of empty space at the bottom of
% pages as new frames are moved to the next page, and if a single frame is too
% long to fit on a page, will completely stop latex from compiling the
% document.  So unless we figure out a solution to this, we'll have to instead
% leave the codecell env. as empty.  I'm keeping the original codecell
% definition here (a thin vertical bar) for reference, in case we find a
% solution to the page break issue.

%\newenvironment{codecell}{%
%     \def\FrameCommand{\color{mediumgray} \vrule width 1pt \hspace{5pt}}%
%    \MakeFramed{\vspace{-0.5em}}}
%  {\unskip\endMakeFramed}

% For now, make this a no-op...
\newenvironment{codecell}{}



 \newenvironment{codeinput}{%
   \def\FrameCommand{\colorbox{inputbackground}}%
   \MakeFramed{\advance\hsize-\width \FrameRestore}}
 {\unskip\endMakeFramed}

\newenvironment{codeoutput}{%
   \def\FrameCommand{\colorbox{outputbackground}}%
   \vspace{-1.4em}
   \MakeFramed{\advance\hsize-\width \FrameRestore}}
 {\unskip\medskip\endMakeFramed}

\newenvironment{code}[1][]{%
   \def\FrameCommand{\colorbox{#1}}%
   \MakeFramed{\advance\hsize-\width \FrameRestore}}
 {\unskip\medskip\endMakeFramed}

\newenvironment{pro}[1][]{%
   \def\FrameCommand{\color[rgb]{0.7,0.95686,1}\vrule width 1mm\normalcolor\colorbox{#1}}%
   \MakeFramed{\advance\hsize-\width \FrameRestore}}
 {\unskip\medskip\endMakeFramed}

% New code to replace ptex2tex:
% Keep old doconce code that creates ptex2tex output
% Add new features to add envirs (like the admon envirs) ++ for
% new code with lstlistings, minted, Verbatim
% the \bpy etc envirs in latex.py, for ptex2tex, are not issued
% if any of the new cml options are used.
% Move code from doconce ptex2tex to various functions for use
% directly by latex.py?
% Make the BlueBar and Blue envirs first, then support minted
% and Verbatim, try to see how these three can have common code.
% Recall: the ptex2tex output for code is very small,
% it's all the preprocess directives that fills up code.
% Maybe all such code could be refactored in lists++ and spit out
% in either plain format or ptex2tex (the admon codes are not easy
% to spot in python, ptex2tex is easier, one sees the alternatives...)
%
%Plan: 1) make Blue and BlueBar, see if it works, then continue as above.

% Use and configure listings package for nicely formatted code
% Good: http://en.wikibooks.org/wiki/LaTeX/Source_Code_Listings (read first)
% http://tex.stackexchange.com/questions/8230/what-configuration-do-you-propose-for-listings-sty-to-make-the-output-look-comfo
%  \begin{lstlisting}[style=python]  % [style=fortran]
%
% Challenge: combining the shadowblueboxes of BlueBar with lstlistings
% a la ipynb, should be straightforward
% listings also has frames and backgrounds
%
% get rid of ptex2tex by having some listing/framed options to play
% around with + minted + plain Verbatim, all of them can be inside
% a colored frame (one should use the stable framed package, not
% mdframed if mdframed features are not required, see
% http://tex.stackexchange.com/questions/71554/framed-or-mdframed-pros-cons
% as of Se, 2012, but it seems that mdframed is what doconce uses and the future

% use escapeinside= to allow for latex commands in the code,
% has to be removed in non-latex envirs
%
% define custom styles with different colors etc for listings
% (mimic ipynb, anslistings), define a set of colors too so
% that just a few parameters govern the choice of style
%
% find a way to handle settings of parameters: \lstset in
% listings vs all the [...] args in minted and Verbatim,
% maybe minted and verbatim have some set-like command too :-)
% parameters are set by command-line args
%
% ALL THIS WILL BE MUCH EASIER TO USE THAN ptex2tex envirs,
% we can just have some preferred combinations of parameters
% (maybe as latex_code_style=name)

\usepackage{listingsutf8}
%\lstset{
%  %language=python,
%  inputencoding=utf8x,
%  extendedchars=\true,
%  aboveskip=\smallskipamount,
%  belowskip=\smallskipamount,
%  xleftmargin=2mm,
%  breaklines=true,
%  basicstyle=\small \ttfamily,
%  showstringspaces=false,
%  keywordstyle=\color{blue}\bfseries,
%  commentstyle=\color{myteal},
%  stringstyle=\color{darkgreen},
%  identifierstyle=\color{darkorange},
%  columns=fullflexible,  % tighter character kerning, like verb
%}

\lstdefinestyle{redblue}{
  inputencoding=utf8x,
  extendedchars=\true,
  aboveskip=\smallskipamount,
  belowskip=\smallskipamount,
  xleftmargin=2mm,
  breaklines=true,
  basicstyle=\small \ttfamily,
  showstringspaces=false,
  keywordstyle=\color{blue}\bfseries,
  commentstyle=\color{myteal},
  stringstyle=\color{darkgreen},
  identifierstyle=\color{darkorange},
  columns=fullflexible,  % tighter character kerning, like verb
}

% The hyperref package gives us a pdf with properly built
% internal navigation ('pdf bookmarks' for the table of contents,
% internal cross-reference links, web links for URLs, etc.)
\usepackage{hyperref}
\hypersetup{
  breaklinks=true,  % so long urls are correctly broken across lines
  colorlinks=true,
  urlcolor=blue,
  linkcolor=darkorange,
  citecolor=darkgreen,
  }

% hardcode size of all verbatim environments to be a bit smaller
\makeatletter
\g@addto@macro\@verbatim\small\topsep=0.5em\partopsep=0pt
\makeatother

% Prevent overflowing lines due to urls and other hard-to-break entities.
\sloppy




\begin{document}



\[\renewcommand{\vec}{\mathbf}\]
\section{Test}


We use the method of manufactured solutions to get an indication of
whether we have implemented the PDE correctly or not.

\begin{codecell}


%\begin{code}[inputbackground]
\begin{pro}[inputbackground_1]
%\begin{codeinput}
% pycod=yellow,lst:redblue
% pycod=blue,minted
\begin{lstlisting}[language=python,style=redblue]
from sympy import *
x, t, rho, dt = symbols('x[0] t rho dt')

def a(u):
    return 1 + u**2

def u_simple(x, t):
    return x**2*(Rational(1,2) - x/3)*t

# MMS: full nonlinear problem
u = u_simple(x, t)
f = rho*diff(u, t) - diff(a(u)*diff(u, x), x)
print ccode(f.simplify())
\end{lstlisting}
\end{pro}
%\end{code}
%\end{codeinput}

\begin{code}[outputbackground]


\begin{Verbatim}[commandchars=\\\{\}]
-rho*pow(x[0], 3)/3 + rho*pow(x[0], 2)/2 + 8*pow(t, 3)*pow(x[0], 7)/9
- 28*pow(t, 3)*pow(x[0], 6)/9 + 7*pow(t, 3)*pow(x[0], 5)/2 -
5*pow(t, 3)*pow(x[0], 4)/4 + 2*t*x[0] - t
\end{Verbatim}

\end{code}

\end{codecell}

And now:

\begin{codecell}


\begin{codeinput}
\begin{lstlisting}
u_1 = u_simple(x, t-dt)
f = rho*diff(u, t) - diff(a(u_1)*diff(u, x), x)
print ccode(f.simplify())
\end{lstlisting}
\end{codeinput}
\begin{codeoutput}


\begin{Verbatim}[commandchars=\\\{\}]
-rho*pow(x[0], 2)*(2*x[0] - 3)/6 +
t*pow(x[0], 4)*pow(dt - t, 2)*pow(x[0] - 1, 2)*
(2*x[0] - 3)/3 + t*(2*x[0] - 1)*
(pow(x[0], 4)*pow(dt - t, 2)*pow(2*x[0] - 3, 2) + 36)/36
\end{Verbatim}

\end{codeoutput}

\end{codecell}


\begin{codecell}


\begin{codeinput}
\begin{lstlisting}
import sys, os
import numpy as np

def bumpy_road(url=None, m=60, b=80, k=60, v=5, cy=False):
    """
    Solve model for verticle vehicle vibrations.

    =========   ==============================================
    variable    description
    =========   ==============================================
    url         either URL of file with excitation force data,
                or name of a local file
    m           mass of system
    b           friction parameter
    k           spring parameter
    v           (constant) velocity of vehicle
    cy          Cython version of function solver
    Return      data (list) holding input and output data
                [x, t, [h,a,u], [h,a,u], ...]
    =========   ==============================================
    """
    # Download file (if url is not the name of a local file)
    if url.startswith('http://') or url.startswith('file://'):
        import urllib
        filename = os.path.basename(url)  # strip off path
        urllib.urlretrieve(url, filename)
    else:
        # Check if url is the name of a local file
        filename = url
        if not os.path.isfile(filename):
            print url, 'must be a URL or a filename'
            sys.exit(1)

    # Load file data into array h_data
    try:
        h_data = np.loadtxt(filename)  # read numpy array from file
    except ValueError:
        print 'Wrong format in file', url
        sys.exit(1)

    x = h_data[0,:]                # 1st column: x coordinates
    h_data = h_data[1:,:]          # other columns: h shapes

    t = x/v                        # time corresponding to x
    dt = t[1] - t[0]
    if dt > 2/np.sqrt(k/float(m)):
        print 'Unstable scheme'

    if cy:
        from solver_cy import solver, Spring
        s = Spring(k)
    else:
        from solver import solver

        def s(u):
            return k*u

    data = [x, t]      # key input and output data (arrays)
    for i in range(h_data.shape[0]):
        h = h_data[i,:]            # extract a column
        a = acceleration(h, x, v)
        F = -m*a

        u, t = solver(I=0, V=0, m=m, b=b, s=s, F=F, t=t,
                      damping='linear')
        data.append([h, F, u])
    return data

def acceleration(h, x, v):
    """Compute 2nd-order derivative of h."""
    # Method: standard finite difference aproximation
    d2h = np.zeros(h.size)
    dx = x[1] - x[0]
    for i in range(1, h.size-1, 1):
        d2h[i] = (h[i-1] - 2*h[i] + h[i+1])/dx**2
    # Extraplolate end values from first interior value
    d2h[0] = d2h[1]
    d2h[-1] = d2h[-2]
    a = d2h*v**2
    return a

def acceleration_vectorized(h, x, v):
    """Compute 2nd-order derivative of h. Vectorized version."""
    d2h = np.zeros(h.size)
    dx = x[1] - x[0]
    d2h[1:-1] = (h[:-2] - 2*h[1:-1] + h[2:])/dx**2
    # Extraplolate end values from first interior value
    d2h[0] = d2h[1]
    d2h[-1] = d2h[-2]
    a = d2h*v**2
    return a

def prepare_input():
    url = 'http://hplbit.bitbucket.org/data/bumpy/bumpy.dat.gz'
    m = 60
    k = 60
    v = 5
    try:
        b = float(sys.argv[1])
    except IndexError:
        b = 80  # default
    return url, m, b, k, v

def command_line_options():
    import argparse
    parser = argparse.ArgumentParser()
    parser.add_argument('--m', '--mass', type=float,
                        default=60, help='mass of vehicle')
    parser.add_argument('--k', '--spring', type=float,
                        default=60, help='spring parameter')
    parser.add_argument('--b', '--damping', type=float,
                        default=80, help='damping parameter')
    parser.add_argument('--v', '--velocity', type=float,
                        default=5, help='velocity of vehicle')
    parser.add_argument('--cython', action='store_true')
    url = 'http://hplbit.bitbucket.org/data/bumpy/bumpy.dat.gz'
    parser.add_argument('--roadfile', type=str,
              default=url, help='filename/URL with road data')
    args = parser.parse_args()
    # Extract input parameters
    m = args.m; k = args.k; b = args.b; v = args.v
    url = args.roadfile; cy = args.cython
    return url, m, b, k, v, cy

if __name__ == '__main__':
    #url, m, b, k, v = prepare_input()
    url, m, b, k, v, cy = command_line_options()

    data = bumpy_road(url=url, m=m, b=b, k=m, v=v, cy=cy)

    # Root mean square values
    u_rms = [np.sqrt((1./len(u))*np.sum(u**2))
             for h, F, u in data[2:]]
    print 'u_rms:', u_rms
    print 'Simulated for t in [0,%g]' % data[1][-1]

    # Save data list to file
    import cPickle
    outfile = open('bumpy.res', 'w')
    cPickle.dump(data, outfile)
    outfile.close()
\end{lstlisting}
\end{codeinput}

\end{codecell}



\end{document}

